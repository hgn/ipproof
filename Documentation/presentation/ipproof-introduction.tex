\documentclass[xcolor=pdftex,dvipsnames,table]{beamer}

\hypersetup{%
	pdfpagemode=FullScreen,%
	pdfstartpage=1,%
	pdfauthor={Hagen Paul Pfeifer},%
}

\usepackage{colortbl}
\usepackage{fancyvrb}
\usepackage{xcolor}
\usepackage{algorithm2e}

\usetheme{Protocollabs}

\useinnertheme{circles}
\usecolortheme{seahorse}
\usecolortheme{rose}


\definecolor{plblue}{RGB}{51,102,153}
\setbeamercolor{structure}{fg=plblue}

\author{Hagen Paul Pfeifer}
\title{IPProof --- A Generic Network Protocol Packet Generator and Behavior Modeler}
\date{\today}

\begin{document}

%%%%%%%%%%%%%%%%%%%
\begin{frame}
\titlepage
\end{frame}




%%%%%%%%%%%%%%%%%%%
\begin{frame}
\frametitle{}
\textsf{There is no one-line answer to the question ''How fast can TCP go?'' -- RFC 1323}
\end{frame}




%%%%%%%%%%%%%%%%%%%
\begin{frame}
\frametitle{Introduction}
\begin{itemize}
	\item Core features of ipproof (20\%)
	\item How to utilize and possible application (80\%)
\end{itemize}
\end{frame}


%%%%%%%%%%%%%%%%%%%
\begin{frame}
\frametitle{Supported Protocols}
\begin{itemize}
	\item Network Layer
	\bi
		\item IPv4
		\item IPv6
	\ei
	\item Transport Layer
	\bi
		\item TCP
		\item UDP
		\item UDP-Lite (patch available)
	\ei
	\item Ipproof can be used to model application level behavior
\end{itemize}
\end{frame}



%%%%%%%%%%%%%%%%%%%
\begin{frame}
\frametitle{Additional Properties}
\begin{itemize}
	\item ipprov provides no analysis functionality
	\item python, ruby, shell scripts in association with tcpdump or pcap traces are required to perform sophisticated analysis,
	ipprov form a vanilla packet generator - not more
	\item Visualization via gnuplot, matplotlib, CairoPlot, octave, R, \dots
\end{itemize}
\end{frame}


%%%%%%%%%%%%%%%%%%%
\begin{frame}
\frametitle{Other Features}
\begin{itemize}
	\item Validation of packet data integrity to check for bit errors (via hamming distance, only payload - no network, transport
	layer header validation)
	\item Windows port (Visual Studio Project file)
	\bi
		\item No UDP-Lite support and other disadvantages due to antiquated network stack
	\ei
\end{itemize}
\end{frame}



%%%%%%%%%%%%%%%%%%%
\begin{frame}
\frametitle{Fundamental Characteristic}
\begin{itemize}
	\item Bulk Data Emulation
	\item Interactive Emulation
	\item Somewhere in between
	\item Bulk Data
	\bi
		\item unidirectional communication characteristic
		\item Sender transmit data
		\item maximum MSS
		\item TCP congestion mechanism are operative (CWND, SSTRESH, ...)
		\item Receiver is limited to acknowledge data
	\ei
	\item Interactive Data
	\bi
		\item bidirectiol communication characteristic
		\item Sender transmit data, receiver "echo's" the origina data
		\item Small amount of data, a few bytes per packet -  far from link capacity
		\item Nagle algorithm disabled
		\item larger delay between successive packets (user types slower then link delay)
	\ei
	\item Somewhere in Between
	\bi
		\item Send n byte every m microseconds, echoed back after p microseconds q bytes or nothing is echoed
		\item Send 1GB nonrecurring without echo
		\item Send 10MB nonrecurring with a 10kb echo packet (e.g. form some kind of application level acknowledgement)
		\item Send 50 byte data, recuring every minute via UDP to a multicast address
	\ei
\end{itemize}
\end{frame}


%%%%%%%%%%%%%%%%%%%
\begin{frame}
\frametitle{Packet Interarrival Modeling}
\begin{itemize}
	\item Network protocol behavior can be divided into two categories concerning the packet interarrival:
	\bi
		\item Constant arrival time, barely deviations from the packet generation process (e.g. streaming application)
		\item Variable packet generation
	\ei
	\item Last but not least: often the packet arrival differs based on ``environment noise'':
	\bi
		\item Runtime environments with Garbage Collections
		\item Operating system process scheduler latency (e.g. high priority tasks versus low priority tasks)
		\item Egress queuing characteristics (e.g. concurrent data streams, rate limiting egress queueing policy)
		\item Middlebox queueing
		\item Network adapter noise (interrupt moderation, \dots)
		\item Ingress queue
		\item Socket buffer characteristics
		\item Operating system scheduling latency
		\item \dots
	\ei
\end{itemize}
\end{frame}



%%%%%%%%%%%%%%%%%%%
\begin{frame}
\frametitle{Receive Buffer Exhausting}
\begin{itemize}
	\item Behaves like an application which forget to read() from the socket (e.g. slow receiver, low priority of receiver
	\item Test to trigger the behavior of zero window probing mechanism of the operating system
\end{itemize}
\end{frame}


%%%%%%%%%%%%%%%%%%%
\begin{frame}
\frametitle{Traffic Models}
\begin{itemize}
	\item Long-lived FTP Traffic
	\item Short-lived Web Traffic
	\item Streaming Video Traffic
	\item Interactive Voice Traffic
\end{itemize}
\end{frame}


%%%%%%%%%%%%%%%%%%%
\begin{frame}
\frametitle{Modelling Interactive Traffic}
\begin{itemize}
	\item Definition: Interactive TCP Traffic
	\bi
		\item TCP/IP illustrated: The protocols, by W. Richard Stevens and Gary R. Wright
		\item IETF Draft "An NS2 TCP Evaluation Tool" provides some metrics about interactive
		traffic\footnote{http://tools.ietf.org/html/draft-irtf-tmrg-ns2-tcp-tool-00}
		\item Examples: SSH, Telnet, IRC, XMPP (Jabber)
	\ei
\end{itemize}
\end{frame}




\end{document}



% vim600: fdm=marker tw=130 sw=4 ts=4 sts=4 ff=unix noet:
